\documentclass{standalone}
\usepackage{tikz}
\usepackage[europeanresistors,americaninductors]{circuitikz}
\usetikzlibrary{chains}
\usepackage{amssymb}
\usepackage{rotating}
\newcommand*{\field}[1]{\mathbb{#1}}
\renewcommand*{\familydefault}{\sfdefault}
\begin{document}
	\begin{sideways}
		\begin{turn}{270}
			\def\firstwidth{4cm}
			\begin{tikzpicture}[
			start chain=going right,
			box/.style={
				on chain,join,draw,
				minimum height=2cm,
				text centered,
				minimum width=\firstwidth,
			},
			every join/.style={ultra thick, ->},
			node distance=5mm
			]
			
			\node [on chain] {Daten}; % Chain starts here
			
			\node [box,xshift=1.5cm,label=above:Typumwandlung, text width=3.9\firstwidth] (rec) {
				double precision $\rightarrow$ 32-bit floating point 
			};
			
			\node [on chain,join,draw, 
			text width=3.9\firstwidth,
			minimum size=width=3.9\firstwidth,
			minimum height=2cm,
			label=above:Filter mit endlicher Impulsantwort,
			] (ic) {
				% links unten recht obentrim = 50mm 165mm 10mm 20mm,
				$y[n] = \sum_{i=0}^N b_{i} \cdot x[n-i]$
			};
			
			\node [box,label=above:Fast Fourier Transformation, text width=3.9\firstwidth, minimum height=2cm] (inv) {
				$\hat{a}_{k} = \sum_{j=0}^{N-1} e^{-2\pi i\cdot\frac{jk}{N}}\cdot a_{j}$ \newline
				
				$k=0,...,N-1$
			};
			
			\node [on chain,join,xshift=10mm, text width=1.5cm]{kontinuierliches \newline Frequenzspektrum};
			% Chain ends here
			
			% CU box
			\node [
			rectangle,draw,
			below=5mm of ic,
			minimum width=15cm,
			minimum height=1cm,
			] (cu) {\textbf{GPU Memory}};
			
			% PU box
			\node [
			rectangle,draw,
			above=2mm of cu,
			minimum width=15cm,
			minimum height=3.2cm
			] (pu) {};
			
			% Connections between CU and PU
			\draw[<->] (rec.south) -- ++(0,-5mm);
			\draw[<->] (cu.north) to (ic.south);
			\draw[<->] (inv.south) -- ++(0,-5mm);
			%% Dummy node so the pictures does not get to small.
			\node[] at (0,-3)(D){};
			\node[] at (20.5,-3)(T){};
			\end{tikzpicture}
		\end{turn}
	\end{sideways}
\end{document}